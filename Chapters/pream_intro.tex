\chapter{Préambule}\label{ch:pream_intro}


\section{Introduction Générale}

\subsection{L'environnement sonore}

Émuler la capacité humaine à percevoir et analyser les objets de l'environnement à partir de données sensibles brutes est aujourd'hui un défi central de l'informatique appliquée. Dans ces recherches, le domaine de l'image s'est taillé la part du lion, stimulé par les nombreuses applications possibles, tant dans le domaine de la recherche d'information que dans celui de la robotique -- avec aujourd'hui l'apparition des premiers véhicules intelligents, capables d'analyser leur environnement. La sphère sonore, pour sa part, a d'abord vu se développer les recherches portant sur les objets spécifiques que sont la voix, puis la musique: chacun bénéficie de champs de recherche dédiés, que ce soit en perception ou en intelligence artificielle (SP: \emph{Speech Processing}, traitement automatique de la parole, et MIR: \emph{Music Information Retrieval}, recouvrement de l'information musicale). Plus récemment, suivant un intérêt croissant pour les problématiques de nuisance sonore, d'une part, et de prise en compte du contexte (\emph{context awareness}), de l'autre, la question de l'analyse d'environnements sonores, c'est-à-dire de scènes sonores "ordinaires" ne relevant ni de la parole, ni de la musique a gagné en importance. C'est dans ce domaine que se positionne le travail présenté dans cette thèse.

On peut tout d'abord remarquer que cette définition des sons environnementaux\footnote{Dans ce document, par souci rédactionnel, nous parlerons indifféremment de son(s) environnemental(aux), d'environnement(s) sonore(s), de scène(s) sonore(s), et de scène(s) sonore(s) environnementale(s), pour désigner les sons environnementaux.} par exclusion -- ni parole, ni musique -- n'est pas satisfaisante. D'un côté, elle réduit les sons environnementaux à des entités secondaires. D'un autre côté, l'opposition suggérée entre sons environnementaux et sons de paroles et/ou de musique, deux domaines où le sens donné aux sons est de première importance, peut mener à penser que l'influence de la valeur sémantique des sons environnementaux est anecdotique, induisant, \emph{de facto}, la primauté de leurs caractéristiques physiques. Ce qui n'est pas apparemment, pas le cas \citep{ballas1987interpreting}.

Nous préfèrerons ici la définition donnée par \cite{vanderveer1980ecological} (cité par \cite{ballas1987interpreting}) qui se pose en quatre points. Un son environnemental:

\begin{enumerate}
\item est produit par une source réelle;
\item a un sens, en vertu de l'action qui en est la cause;
\item est par essence plus complexe qu'un stimuli de synthèse produit en laboratoire, comme un son pur;
\item ne fait pas partie d'un système de communication.
\end{enumerate}

Les deux premiers points caractérisent directement les sources émettrices, précisant qu'il s'agit de sources réelles, et insistant sur l'importance du sens qu'elles portent. Nous remarquons cependant que la définition pose la valeur sémantique des sources uniquement par rapport à l'action à l'origine du son. Or l'effet du contexte, et notamment celui relatif à l'individu récepteur, est de première importance: une même scène sonore peut être perçue différemment par deux individus. Il nous paraît donc important de renforcer le point deux de la définition comme suit:

\begin{itemize}
\setcounter{enumi}{2}
\item a un sens, en vertu de l'action qui en est la cause, ainsi que du contexte d'écoute.
\end{itemize}

Les deux derniers points positionnent les sons environnementaux par rapport aux autres stimuli sonores couramment étudiés, les opposant spécifiquement aux sons de synthèse produits en laboratoire, ainsi qu'aux sons ayant une portée communicationnelle comme la parole ou la musique.

La définition insiste sur le fait que la perception d'un environnement sonore dépend éminemment de l'interprétation sémantique des événements qui le peuplent, \ie~de l'identification de la nature des sources sonores émettrices. Cette importance de la composition sémantique sur les qualités sensibles des scènes nous permet ainsi d'envisager la scène comme un objet composite, le résultat de l'association des sources sonores qui la constituent.

Partant de cette vision composite des scènes, l'objectif de nos travaux est triple:

\begin{itemize}
\item proposer un modèle morphologique des scènes sonores environnementales, fondé sur une étude approfondie de la littérature ayant trait aux mécanismes régissant la perception des sons environnementaux;
\item montrer l'utilité d'un tel modèle dans le cadre de l'analyse sensorielle;
\item montrer l'utilité d'un tel modèle dans le cadre de l'analyse automatique;
\end{itemize}

\subsection{Pourquoi modéliser une scène sonore ?}

Que ce soit dans le domaine de la perception ou de l'apprentissage machine, toute expérience suppose un niveau de contrôle maximal de l'expérimentateur sur les caractéristiques des stimuli proposés. En ce qui concerne les environnements sonores, peu de travaux ont porté sur le développement d’outils pouvant permettre aux chercheurs d'agir sur ces stimuli.

Conscient de cette problématique, nous proposons ici un modèle génératif permettant de simuler, à partir d'enregistrements de sons isolés, des scènes sonores dont nous maîtrisons les propriétés structurelles, à savoir, l'intensité, la densité et la diversité des sources sonores en présence. Le modèle envisage la scène sonore comme un objet composite, une somme de sons sources. Le niveau d'abstraction choisi est motivé par les connaissances disponibles sur le système auditif humain.

Fort des banques de données ainsi constituées, nous investissons deux champs d'application. Le premier concerne la perception des paysages sonores, et questionne plus spécifiquement la notion d'agrément perçu dans des lieux urbains. L'utilisation de données simulées nous permet d'apprécier finement les contributions de chacune des sources sonores dans l'agrément perçu. Elle nous permet encore de retravailler les scènes en modifiant les paramètres afin d'en mesurer les effets.

Le deuxième concerne la détection automatique des événements sonores, et propose une méthodologie novatrice afin d'évaluer les performances des algorithmes dédiés à cette tâche. Les données simulées se révèlent un outil précieux afin d'évaluer la capacité de généralisation des algorithmes.

% \gl{Suggérer les résultats que l’on  peut avoir avec la simulation et pas avec le reste}

\section{Motivations des cas d'études}

\subsection{Un cadre applicatif pluridisciplinaire}

Comme précédemment évoqué, l'application des données simulées issues de notre modèle d’environnements sonores porte à la fois sur l'analyse sensorielle, et sur l'analyse automatique des environnements.

Par analyse sensorielle, on entend l'ensemble des processus qui constituent le système perceptif de l'homme, système par lequel il comprend son environnement, lui donne sens. Ces processus comprennent, d'une part, les mécanismes d'acquisition de l'information, d'autre part, les mécanismes de traitement de l'information.

Par analyse automatique on entend l'apprentissage machine. Dans ce domaine, l'objectif des recherches est d'élaborer des algorithmes permettant à une machine de mimer la perception humaine. Ici encore on distingue les étapes d'acquisition de l'information, et de traitement de l'information. Les études portant sur l'acquisition se focalisent sur les descripteurs mathématiques permettant d'extraire une information utile des données brutes du signal. Les études portant sur le traitement se penchent sur les techniques permettant de trier l'information ainsi collectée, c’est-à-dire regrouper les parties de l'information qui se ``\,ressemblent\,''. Ce tri peut s'opérer soit dans un cadre non-supervisé, \ie~en effectuant les groupements sur la seule base de l'information collectée, soit dans un cadre supervisé, \ie~en effectuant les groupements sur la base de classes d'objets pré-considérées.

L'analyse sensorielle a donc trait à la perception humaine, et l'analyse automatique à l'intelligence artificielle. Les deux domaines peuvent, à première vue, paraître éloignés. Ils portent cependant sur l'acquisition, la structuration et l'utilisation des connaissances, et constituent les deux disciplines d'une même quête, \ie~la pensée humaine, l'une visant à comprendre son fonctionnement, l'autre cherchant à le simuler. À ce titre, perception humaine et intelligence artificielle font toutes deux partie d'un même champ de recherche: les sciences cognitives.

\subsection{La perception des paysages sonores urbains}

\subsubsection{Historique et application}
\label{sec:ch3_urbanNoiseSoundscape}

La ville est un environnement bruyant. Elle l'a été de tous temps. Déjà dans les rues de la Rome antique, le bruit des chariots pose problème\footnote{\cf~Juvenal, Satire 3.232–238}. Le consul Jules César interdit d'ailleurs à ceux-ci de circuler la nuit. Ce qui a changé, par contre, c'est la perception du bruit. Dans les années 70-80, le bruit ``\,devient\,'' pollution, facteur de dégradation de la qualité de vie. Cette pollution est d'autant plus critique que d'ici 2050, 68\% de la population mondiale sera urbaine \citep{park14}.

Les chercheurs se concentrent alors sur l'identification des sources du bruit, et sur les moyens d'abaisser les niveaux sonores. Les premières législations anti-bruit apparaissent, qui proposent/imposent une réduction du niveau des bruits produits, essentiellement, par les transports et l'industrie.

Mais le problème persiste, le bruit demeurant un phénomène subjectif, autrement dit, dépendant de l'appréciation de l'auditeur. Le bruit est affaire de contexte, et beaucoup de lieux urbains sont appréciés aussi pour leur atmosphère vivante, c'est à dire ``\,bruyante\,''. Ville agréable ne rime pas nécessairement avec ville silencieuse.

Il est aujourd'hui communément admis que des mesures objectives de niveaux sonores (\eg~$L_{Aeq}$) ne peuvent, seules, rendre compte de la qualité d'un environnement, et que vouloir réhabiliter cet environnement en s'attaquant uniquement aux paramètres acoustiques, par définition objectifs, est illusoire \citep{yang2005acoustic,schulte2006soundscape,kang2010semantic,aletta2016soundscape}. Il faut désormais envisager le bruit non plus seulement comme un objet physique, mais encore comme un objet cognitif \citep{guastavino_etude_2003}. Le problème n'est plus de savoir à partir de quand un son est gênant, mais pourquoi il est perçu comme tel, et par tel individu.

C'est l'objet de la recherche sur les paysages sonores: envisager l'environnement sonore du point de vue de celui qui le perçoit.

Centrées sur le sujet, ces recherches sont par essence interdisciplinaires \citep{davies2013perception,aletta2016soundscape}, faisant appel à des outils et des méthodes provenant de champs de recherche variés comme l'acoustique, la psychologie cognitive, la psycho-linguistique, la sociologie, et plus récemment, l’intelligence artificielle.

Depuis vingt ans, l'approche par les paysages sonores a permis de développer une base de descripteurs qualitatifs et acoustiques grâce auxquels nous jugeons mieux, et sommes mieux à même d'améliorer l'environnement sonore urbain. \citep{kang2006urban,schulte2007soundscape}

L'enjeu est aujourd'hui de relier ces données perceptives à des mesures acoustiques, afin de pouvoir établir une politique de réduction du bruit efficace, adaptée à chaque situation \citep{schulte2013soundscape}.

Cependant, le caractère pluridisciplinaire de ces recherches, et l'utilisation de protocoles expérimentaux variés pour évaluer l'environnement sonore, rendent l’intégration des résultats difficile \citep{davies2013perception}. De plus, il n'y a toujours pas de consensus sur les descripteurs (acoustiques ou perceptifs) à utiliser pour caractériser un paysage sonore \citep{brocolini2012prediction,aletta2016soundscape}, ce qui empêche la communauté, d'une part, de présenter aux décideurs des indicateurs génériques d'évaluation des paysages sonores, et d'autre part, d'élaborer/proposer des modèles crédibles sur la base de ces expertises.

Récemment, plusieurs projets internationaux ont été lancés afin de standardiser les pratiques expérimentales des recherches portant sur les paysages sonores, notamment \emph{the European Cooperation in Science and Technology Action}\footnote{TD0804: \emph{soundscape of European Cities and Landscapes}: \url{http://www.cost.eu/COST_Actions/tud/TD0804}} \citep{schulte2010soundscape} et \emph{the Positive Soundscape project} \citep{salford2106,davies2013perception}. Mais les difficultés persistent \citep{schulte2013soundscape,ribeiro2013heart}. Afin d'acquérir la masse de données nécessaire pour évaluer la qualité de l'environnement sonore sur un temps long, les caractéristiques de l'environnement variant au cours de la journée, comme au cours des saisons, d'autres projets \citep{park14} ont pour objet le déploiement d'un réseau de senseurs capable de capturer, en continu et en temps réel, toutes les informations relatives à la qualité de l’environnement évalué.

\subsubsection{Évaluation perceptive des paysages sonores}

\textbf{Une description holistique insatisfaisante}\\

Un des objectifs premiers des études sur les paysages sonores est d'identifier les informations contenues dans l'environnement qui influent sur les sensations perçues. La majorité des études considèrent des descriptions holistiques, calculant des indicateurs (physiques ou perceptifs) sur l'ensemble de l'environnement.

Cependant, de plus en plus de travaux tendent à montrer que toutes les sources ne participent pas de manière égale à la perception de la scène sonore. En conséquence, les recherches se portent maintenant sur l'étude des contributions spécifiques des différents éléments qui composent l'environnement.

Ces recherches peuvent bénéficier de l'utilisation de données simulées, données dont les caractéristiques structurelles, en particulier la nature des sources présentes, leurs positions, et leurs caractéristiques physiques, sont connues. \\

{\setlength{\parindent}{0cm}\textbf{Un accès partiel à la représentation mentale du paysage sonore}} \\

Questionner le lien entre sensation (\eg~calme) et environnement (\eg~scène sonore urbaine), revient à objectiver la représentation mentale que se fait un sujet d'une scène sonore en particulier (\eg~une scène sonore urbaine calme).

Si on demande au sujet d'évaluer un environnement donné (évaluez le ``\,calme\,'' de cette scène sonore), le gain d'information de l'expérimentateur est conditionné par la nature des stimuli disponibles, que ces derniers soient des scènes sonores enregistrées (expérience en laboratoire), ou réelles (expérience \emph{in situ}). Cependant, le fait que ces stimuli existent lui permet d'en établir une description physique précise.

Si, à l'inverse, on demande au sujet de décrire un environnement donné (décrivez une scène sonore urbaine ``\,calme\,''), l'expérimentateur obtient bien une information symbolique, sémantique, et riche de la représentation que se fait le sujet de cet environnement, mais en l'absence de données sonores, il ne peut la caractériser physiquement.

Nous pensons que la simulation permet de faire un lien élégant entre ces deux approches auparavant disjointes. Celle-ci peut être vue comme une description modale, \ie~définie sur des dimensions physiques (le signal de la scène simulée), de la représentation mentale du sujet, description modale dont il est possible d'extraire les caractéristiques physiques.


\subsection{La détection automatique d'événements sonores}
\label{ch1_SED}

\subsubsection{Historique et application}

La quantité de données audio enregistrées à partir de notre environnement sonore a considérablement augmenté au cours des dernières décennies. Afin de mesurer l'effet de l'activité humaine et du changement climatique sur la biodiversité du règne animal, les chercheurs travaillant dans le domaine de l'eco-acoustique \citep{ecoacoustic2014,krause} ont récemment entrepris le déploiement massif de capteurs acoustiques à travers le monde \citep{warren2006urban,nessSST13,stowell13a,stowell13b}.

Par ailleurs, des travaux récents démontrent l'intérêt de monitorer l'environnement acoustique des zones urbaines afin d'en caractériser l'agrément  \citep{guyot2005urban,ricciardi2015sound}, ainsi que la gêne due aux bruits de circulation \citep{gloaguen2016}. De par leur impact sociétal, et du fait des nombreux défis scientifiques qu'elles soulèvent, l'intérêt de ces études est majeur.

Afin de répondre aux problématiques soulevées par les recherches ci-dessus exposées, l'Analyse Automatique de Scènes Sonores (A$^2$S$^2$) \citep{Stowell15} vise à développer des approches et des systèmes permettant d'extraire automatiquement des environnements sonores une information utile.

On distingue alors deux tâches dans l'A$^2$S$^2$:

\begin{itemize}
\item la classification des scènes acoustiques (ASC: \emph{acoustic scene classification}), dont l'objectif est de reconnaître automatiquement le type d'environnement (parc, rue calme, marché) d'un enregistrement donné;
\item la détection d'événements sonores (SED: \emph{sound event detection}), dont l'objectif est de détecter et d'identifier les différents événements qui composent l'enregistrement d'un environnement donné (parc $\Rightarrow$ chant d'oiseau, ballon, bruit de pas, voix).
\end{itemize}

C'est de la dernière tâche que nous allons traiter.

\subsubsection{Évaluation des algorithmes de détection d'événements}

Si la reconnaissance automatique de la parole \citep{Rabiner93}, ou encore le recouvrement automatique de l'information musicale \citep{Muller07} sont des domaines d'investigation aujourd'hui bien établis, l'A$^2$S$^2$ en est à ses balbutiements, et ne profite pas encore des banques de données suffisantes, en volume et en qualité, ce qui contraint l'effort de recherche.

Ces banques de données sont cependant essentielles pour obtenir une évaluation saine et informative des systèmes proposés. Une mauvaise conception de ces dernières peut mener à des erreurs d’appréciation des systèmes. 

Ce fait a déjà été constaté dans les domaines liés à l’analyse automatique de la musique \citep(sturm2014simple), mais également dans le cadre de nos travaux, en ce qui concerne le recouvrement de similarités entre scènes sonores environnementales \citep(lafay2016JAES).

\gl{Dans le cas de l’évaluation des algorithmes de détection d'événements sonores, il nous apparaît donc important de considérer des corpus de scènes dont les caractéristiques structurelles sont maîtrisées par l'expérimentateur.} 

%\gl{TODO: cette dernière précision tombe un peu comme un cheveu dans la soupe, jusque là on traitait de la qualité et du volume des banques de données. Si parler maintenant de la maîtrise des caractéristiques structurelles des scènes, il faut amener le sujet}

\section{Plan}

Ce document comprend 4 parties.

La partie~\ref{part:preamb} est constituée du chapitre~\ref{ch:pream_intro}, la présente introduction.

La partie~\ref{part:model} introduit le modèle morphologique de scènes sonores. Elle regroupe les chapitres~\ref{ch:psycho_ea}, et~\ref{ch:psycho_model}. Le chapitre~\ref{ch:psycho_ea} présente un état de l'art des connaissances sur les processus mis en œuvre par le système auditif dans la perception des environnements sonores. Sur la base de ces considérations perceptives, le chapitre~\ref{ch:psycho_model} présente un modèle morphologique de scènes sonores environnementales. Il introduit également les outils permettant de simuler, à partir du modèle proposé, des environnements sonores.

La partie~\ref{part:simu} présente des cas d'applications concrets pouvant profiter de l'utilisation de données simulées. Elle est composée des chapitres~\ref{ch:psycho_xp}, et~\ref{ch:ml_simuperf}. Le chapitre~\ref{ch:psycho_xp} présente une série d'expériences montrant comment le modèle introduit permet d'étendre les possibilités des méthodologies traditionnellement utilisées en analyse sensorielle. Le cadre applicatif choisi par ces expériences est l'évaluation de l'agrément dans les environnements sonores urbains. Le chapitre~\ref{ch:ml_simuperf} précise comment le modèle de scènes sonores proposé peut être appliqué à l'évaluation des algorithmes de détection automatique d'événements sonores. Il montre notamment comment il permet de gagner en connaissance quant à la capacité de généralisation des systèmes de détection proposés.

La partie~\ref{part:conc}, partie conclusive, comprend le chapitre~\ref{ch:end_conc}, qui résume les différentes contributions de cette thèse, conclut quant au travail effectué, et propose de nouvelles pistes à explorer.

% La présente thèse traite des environnements sonores. Donnons une définition. Habituellement on entend par son environnemental\footnote{Dans ce document, par souci rédactionnel, nous parlerons indifféremment de son(s) environnemental(aux), d'environnement(s) sonore(s), de scène(s) sonore(s), et de scène(s) sonore(s) environnementale(s), pour désigner les sons environnementaux.} tout extrait sonore qui ne se réclame ni de la parole, ni de la musique. Il s'agit alors d'une définition par exclusion, musique et parole étant des stimuli étudiés depuis longtemps, bien plus que leurs pendants environnementaux. Chacun bénéficie de champs de recherche dédiés, que ce soit en perception\footnote{on parle de perception de la parole (\emph{speech perception}) et de perception de la musique (\emph{music perception})}, ou en intelligence artificielle\footnote{On parle de traitement automatique de la parole (SP: \emph{Speech Processing}) et de recouvrement de l'information musicale pour la musique (MIR: \emph{Music Information Retrieval})}.

% Nous pensons que cette définition n'est pas satisfaisante. 




