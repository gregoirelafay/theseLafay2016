%************************************************
\chapter{Introduction}\label{ch:pream_intro}
%************************************************

\jlv{La thèse s'intéresse aux environnements sonores. L'étude se veut pluridisciplinaire, portant à a fois sur l'analyse sensorielle et l'analyse automatique de ces environnements.}
\jls{La présente thèse traite des environnements sonores. L'étude se veut pluridisciplinaire. Elle porte à la fois sur l'analyse sensorielle, et sur l'analyse automatique des environnements.}

\jlv{Par analyse sensorielle, nous entendons l'ensemble des processus qui constituent le système perceptif humain, et lui permettent d'analyser l'environnement qui l'entoure, autrement dit d'en faire sens. Ces processus comprennent à la fois les mécanismes d'acquisition de l'information, et les mécanismes de traitement.} 
\jls{Par analyse sensorielle, on entend l'ensemble des processus qui constituent le système perceptif de l'homme, système par lequel il comprend son environnement, lui donne sens. Ces processus comprennent, d'une part, les mécanismes d'acquisition de l'information, d'autre part, les mécanismes de traitement de l'information.}

\jlv{L'analyse automatique se réfère, quant à elle, au domaine de l'apprentissage machine. L'objectif de ces recherches est d'élaborer des algorithmes permettant à une machine de mimer la perception humaine. Là encore on prend l'habitude de distinguer les étapes d'acquisitions, et de traitement. Les études portant sur l'acquisition de l'information se focalisent sur les descripteurs mathématiques permettant d'extraire une information utile des données brutes du signal. Les études portant sur le traitement se penchent sur les techniques permettant de trier l'information ainsi décrite. Par trie, on entend généralement regrouper les parties de l'information qui se ``\,ressemblent\,''. Ce groupement peut s'opérer soit dans un cadre supervisé, \ie~en considérant une information \emph{a priori} sur la nature des classes d'objets à obtenir, soit dans un cadre non-supervisé, \ie~en effectuant des groupements sur la base seule de l'information à trier.}
\jls{Par analyse automatique on entend l'apprentissage machine. Dans ce domaine, l'objectif des recherches est d'élaborer des algorithmes permettant au "robot" de mimer \jlc{reproduire} la perception humaine. Ici encore on distingue les étapes d'acquisition de l'information, et de traitement de l'information. Les études portant sur l'acquisition se focalisent sur les descripteurs mathématiques permettant d'extraire une information utile des données brutes du signal. Les études portant sur le traitement se penchent sur les techniques permettant de trier l'information ainsi décrite. Trier signifiant regrouper les parties de l'information qui se ``\,ressemblent\,''. Ce tri peut s'opérer soit dans un cadre non-supervisé, \ie~en effectuant les groupements sur la seule base de l'information à collectée, soit dans un cadre supervisé, \ie~en effectuant les groupements sur la base de classes d'objets pré considérées.} \jlc{"sur la base de classes d'objets pré considérées" sur ce coup là je m'avance un peu. Tu peux vérifier?}


\jlv{L'analyse sensorielle a donc trait à la perception humaine, et l'analyse automatique à l'intelligence artificielle. Si ces deux disciplines scientifiques peuvent, de part leurs objectifs et méthodologies différentes, sembler éloignées, elles traitent néanmoins du même problème, à savoir, l'acquisition, la structuration et l'utilisation des connaissances. Ces deux domaines d'études sont les deux faces de la même pièce, \ie~la pensée humaine, la première tentant de comprendre son fonctionnement, et la deuxième de le simuler. À ce titre, elles font toutes deux partie d'un même champ de recherche: les sciences cognitives.} 
\jls{L'analyse sensorielle a donc trait à la perception humaine, l'analyse automatique à l'intelligence artificielle. Les deux domaines peuvent paraître éloignés. Ils portent cependant sur l'acquisition, la structuration et l'utilisation des connaissances, et constituent les deux disciplines d'une même quête, \ie~la pensée humaine, l'une visant à comprendre son fonctionnement, l'autre cherchant à le simuler. À ce titre, perception humaine et intelligence artificielle font toutes deux partie d'un même champ de recherche: les sciences cognitives.}


Les travaux présentés dans cette étude se limitent à une modalité sensorielle particulière, l'audition. Plus particulièrement, elle se focalise sur un type de stimuli: les sons environnementaux. 

Donnons une définition. Habituellement on entend pas son environnemental\footnote{Dans ce document, par souci rédactionnel, nous parlerons indifféremment de son(s) environnemental(aux), d'environnement(s) sonore(s), de scène(s) sonore(s), et de scène(s) sonore(s) environnementale(s), pour désigner les sons environnementaux.} tout extrait sonore qui ne se réclame ni de la parole, ni de la musique. Il s'agit d'une définition par exclusion, musique et parole étant des stimuli étudiés depuis longtemps, bien plus que leur pendant environnementaux. Chacun bénéficie de champs de recherche dédiés, que ce soit en perception\footnote{on parle de perception de la parole (\emph{speech perception}) et de perception de la musique (\emph{music perception})}, ou en intelligence artificielles\footnote{On parle de traitement automatique du langage naturel pour la parole (NLP: \emph{Natural Language Processing}) et de recouvrement de l'information musicale pour la musique (MIR: \emph{Music Information Retrieval})}.

Cette définition \jlc{exit: par la négative} n'est pas satisfaisante. D'un coté, \jlv{elle place les sons environnementaux comme} \jls{elle réduit les sons environnementaux à} des entités secondaires. \jlv{D'un autre coté, l'opposition suggérée entre son environnementaux \vs~sons de paroles et de musiques, deux entités où le sens donné aux sons est de première importance, peut mener à penser que l'influence de la valeur sémantique des sons environnementaux dans leurs traitements est anecdotique, impliquant \emph{de facto} la primauté de leurs caractéristiques physiques} \jls{D'un autre côté, l'opposition suggérée entre sons environnementaux et sons de paroles et/ou de musique, deux domaines où le sens donné aux sons est de première importance, peut mener à penser que l'influence de la valeur sémantique des sons environnementaux est anecdotique, induisant, \emph{de facto}, la primauté de leurs caractéristiques physiques}. Ce qui n'est pas le cas \citep{ballas1987interpreting}. 

Nous prenons dans ce document la définition donnée par \cite{vanderveer1980ecological} (cité par \cite{ballas1987interpreting}).

La définition est en quatre points. Un son environnemental :

\begin{enumerate}
\item est produit par une source réelle;
\item a un sens, en vertu de l'action qui en est la cause;
\item est par essence plus complexe qu'un stimuli de synthèse produit en laboratoire, comme un son pur;
\item ne fait pas partie d'un système de communication.
\end{enumerate} 

Les deux premiers points caractérisent directement les sources émettrices \jlc{exit: des sons environnements}, précisant qu'il s'agit de sources réelles, et insistant sur l'importance du sens qu'elles portent. Nous remarquons cependant que la définition pose la valeur sémantique des sources uniquement par rapport à l'action \jlv{étant la cause le son} \jls{à l'origine du son}. Or l'effet du contexte, et notamment celui relatif à l'individu récepteur, est de première importance: une même scène sonore peut être perçue différemment par deux individus \jlc{exit: différents}. Nous nous proposons ainsi de renforcer le point deux de la définition comme suit:

\begin{itemize}
\setcounter{enumi}{2}
\item a un sens, en vertu de l'action qui en est la cause, ainsi que du contexte d'écoute.
\end{itemize}

Les deux derniers points positionnent les sons environnementaux par rapport aux autres stimuli sonores couramment étudiés, les opposant spécifiquement aux sons de synthèses produits en laboratoire, ainsi qu'aux sons assumant une portée communicationnelle comme la parole ou la musique.

La définition insiste sur le fait que la perception d'un environnement sonore dépend éminemment de l'interprétation sémantique des événements qui la peuplent \jlc{"qui la peuplent" j'aurais dit qui "le" peuplent "le" renvoyant à environnement}, \ie~de l'identification de la nature des sources sonores émettrices. Cette importance de la composition sémantique sur les qualités sensibles des scènes nous permet ainsi d'envisager la scène comme un objet composite, le résultat de l'association des sources sonores qui la constituent.

Partant de cette vision composite des scènes, l'objectif de nos travaux est triple:

\begin{itemize}
\item proposer un modèle morphologique des scènes sonores environnementales, modèle fondé sur une étude approfondie de la littérature ayant trait aux mécanismes régissant la perception des sons environnementaux;
\item montrer l'utilité d'un tel modèle dans le cadre de l'analyse sensorielle;
\item montrer l'utilité d'un tel modèle dans le cadre de l'analyse automatique;
\end{itemize}

Ces objectifs sous-entendent \jls{déterminent} l'organisation de ce document. \jlv{Ce dernier se découpe en quatre parties} \jls{Il comprend 4 parties}. \jlv{La première regroupe le chapitre 1, la présente introduction, et le chapitre 2, qui motive l'approche pluridisciplinaire adoptée dans nos travaux.} \jls{La partie 1 est constituée du chapitre 1, la présente introduction, et du chapitre 2, où est motivée l'approche pluridisciplinaire adoptée dans nos travaux.}

\jlv{La partie 2 traite de l'analyse sensorielle. Elle regroupe trois chapitres.} \jls{La partie 2 traite de l'analyse sensorielle. Elle regroupe les chapitres 3, 4 et 5.} Le chapitre 3 présente un état de l'art des connaissances sur les processus mis en œuvre par le système auditif lors de la perception des environnements sonores. Sur la base de ces considérations perceptives, le chapitre 4 présente un modèle morphologique de scènes sonores environnementales. Il introduit également les outils permettant de simuler, \jlv{à partir modèle proposé} \jls{à partir d'un modèle proposé}, des environnements sonores. Le chapitre 5 présente une série d'expériences montrant comment le modèle introduit permet d'étendre les possibilités des méthodologies classiquement utilisées en analyse sensorielle. Le cadre applicatif choisi par ces expériences est l'évaluation de l'agrément dans les environnements sonores urbains.


\jlv{La partie trois traite de l'analyse automatique des environnements sonores. Elle est composée de trois chapitres.} \jls{La partie 3 traite de l'analyse automatique des environnements sonores. Elle est composée des chapitres 6, 7 et 8.} Le chapitre 6 présente un état de l'art des connaissances sur l'apprentissage machine \jlc{exit: comme} appliqué aux sons environnementaux. Cette présentation s'attache \jlc{s'attache/tâche...} \jls{s'emploie} à détailler les différentes tâches (classification ou détection de sons isolés ou de scènes complexes), les algorithmes couramment utilisés, ainsi que les pratiques expérimentales ayant cours dans ce domaine. Par pratiques expérimentales on entend ici la manière avec laquelle sont évaluées les performances des systèmes proposés, notamment en ce qui concerne les banques de données et les métriques utilisées. Le chapitre 7 précise comment le modèle de scènes sonores proposé peut être appliqué à l'évaluation des algorithmes de détection automatique d'événements sonores. Il montre notamment comment \jlv{ce dernier} \jls{il, ce modèle,...} permet de gagner en connaissance quant à la capacité de généralisation des systèmes de détection proposés. Dans le chapitre 8, nous nous éloignons légèrement du modèle introduit, et proposons un algorithme non-supervisé permettant de recouvrer les similarités existantes entre différentes scènes sonores. Cet algorithme s'appuie sur la vision composite des scènes sonores, en adoptant une approche dite objet. Utilisant une représentation \emph{sparse} du signal, via un descripteur de type \emph{scattering}, l'approche objet groupe, dans un premier temps, \jlv{les éléments d'une même scène étant similaires} \jls{les éléments similaires d'une même scène}. La similarité inter-scène est ensuite calculée sur la base des différents groupes précédemment obtenus. 

\jlv{Enfin, la partie conclusive est composée de deux chapitres: le chapitre 9 qui résume les différentes contributions de cette thèse et conclue quant aux travail effectué. Le chapitre 10 qui discute plus avant à la fois l'approche pluridisciplinaires adoptée et également les résultats obtenus, et propose de nouvelles pistes à explorer.}
\jls{La partie 4, partie conclusive, comprend les chapitres 9 et 10. Le chapitre 9 résume les différentes contributions de cette thèse, et conclue quant au travail effectué. Le chapitre 10, lui, \jlc{"discute plus avant" c'est mal dit. reformuler} à la fois l'approche pluridisciplinaires adoptée et \jlc{exit: également} les résultats obtenus, et propose de nouvelles pistes à explorer.}





 
  
 

%*****************************************
%*****************************************
%*****************************************
%*****************************************
%*****************************************




