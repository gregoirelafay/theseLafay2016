%************************************************
\chapter{Un modèle morphologique des scènes sonores environnementales}\label{ch:psycho_model} % $\mathbb{ZNR}$
%************************************************

\section{Enjeux et motivations}

\section{Formalisation du modèle}

\section{Du modèle à la simulation}

G1: \citep{bruce2009development,bruce2014effects}; G2 \citep{davies2014soundscape} montre que quand on demande à des participants de simuler un paysage sonore, les simulations font références à ce que les participants s'imaginent être un environnement typique, sans tenir compte de leur propre préférence pour des sons particuliers.



%*****************************************
%*****************************************
%*****************************************
%*****************************************
%*****************************************
