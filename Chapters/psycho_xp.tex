%************************************************
\chapter{Application du modèle morphologique à l’étude perceptive des environnements sonores urbains}\label{ch:psycho_xp}
%************************************************

\section{Introduction}

\section{Analyse de la contribution spécifique des sources sonores dans l'agrément}

\subsection{Objectif de l'expérience}

\subsection{Banque de données de sons isolés}

\subsection{Planification expérimentale}

\subsubsection{Planification expérimentale: Épreuve de simulation}

\textbf{Procédure} \\

Les sujets doivent simuler deux environnements sonores urbains, chacun d'une durée de 1 minute.  Pour les deux simulations, les sujets doivent se conformer aux consignes suivantes.

\begin{itemize}
\item Première simulation : Simuler un paysage sonore \textbf{urbain plausible} qui selon vous est idéal (où vous aimeriez vivre).
\item Deuxième simulation : Simuler un paysage sonore \textbf{urbain plausible} qui selon vous est non-idéal (où vous n'aimeriez pas vivre).
\end{itemize}

Tous les sujets commencent par simuler l'environnement idéal. Les sujets ne prennent connaissance de la deuxième consigne qu'à la fin de la première simulation.

Les sujets son totalement libres dans le choix des sons, et des paramètres. Il doivent cependant se conformer à deux contraintes:

\begin{itemize}

\item Les sujets doivent mimer un auditeur statique.
 
\item Les environnements simulées doivent être plausible, \ie~ils ne doivent pas simuler de situation physiquement irréalistes. L'exemple suivant d'une situation physiquement irréaliste est donné aux sujets: 

\begin{quote}
Un chien aboyant toute les 10 millisecondes.
\end{quote}

\end{itemize}

Avant de commencer la première simulation, les sujets doivent réaliser un petit tutoriel de 20 minutes, afin de se familiariser avec le logiciel de simulation.

L'expérience est prévue pour durer 2h30.

\textbf{Apparatus} \\

Tous les sujets passent l'expérience sur des machines identiques (\gl{description des machines}). L'audio est présenter en stéréophonie, par le biais de casques audio. Pendant le tutoriel,les sujets doivent ajuster le niveau sonore à un volume confortable. Ils ne peuvent le modifier par la suite.

Tous les sujets réalisent l'expérience simultanément. Ils sont répartis de manière égale dans trois pièces identiques, toutes possédant un environnement calme. Ils n'ont pas le droit de s'adresser la parole pendant l'expérience.

Trois expérimentateurs, un dans chaque pièce, sont présents durant la totalité de l'expérience, afin de contrôler le bon déroulement de cette dernière, et de répondre à d'éventuelles questions des sujets. 

\textbf{Participants} \\

44 étudiants (14 femmes) de L’École Centrale de Nantes ont participé à l'expérience. Ils ont tous un peu près le même âge (moyenne: 21.6, écart-type: 2). Tous les sujets ont vécu dans la même ville (Nantes), au minimum pendant les deux dernières années précédant l'expérience.

Dû a une incompréhension des consignes, ou à l'impossibilité de finir dans les temps, 4 sujets ne sont pas pris en compte pour l'analyse. 40 sujets ont réalisés l'expérience avec succès, nous fournissant ainsi 80 scènes sonores simulés, 40 idéals, et 40 non-idéals.

\subsubsection{Planification expérimentale: Épreuve d'évaluation de l'agrément}

\textbf{Procédure} \\

Les sujets doivent évaluer l'agrément des 80 scènes sonores simulées. Dû à des contraintes de temps, les sujets n'évaluent que 30 secondes des scènes simulées (à l'origine d'une durée de 1 minute). Ces 30 secondes sont extraites à partir de la 15ème et jusqu'à la 45ème seconde des scènes simulées de 1 minute.

L'évaluation s'effectue sur une échelle sémantique bipolaire de 7 points allant de -3 (non-idéal/très désagréable) à +3 (idéal/très agréable). Avant de noter une scènes, les sujets doivent obligatoirement écouter les 20 premières secondes de cette dernière. Après la notation, ils sont libre de passer à la scènes suivante, avant la fin des 30 secondes.

Pour chaque sujet, les scènes sont présentées dans un ordre aléatoire. Les dix premières scènes permettent au sujet d'ajuster ses notes. Elles sont obligatoirement composées de 5 scènes idéales et 5 non-idéales. Ces dix premières scènes sont rejouées à la fin de l'expérience, et seules les notes données à la deuxième occurrence sont prises en compte.

\textbf{Apparatus} \\

Tous les sujets passent l'expérience sur des machines identiques (\gl{description des machines}). L'audio est présenté en stéréophonie, par le biais de casques audio semi-ouvert \emph{Beyer-Dynamic DT 990 Pro}. Toutes les scènes sonores ont été re-simulée sur la base des partitions (\gl{citer section décrivant le terme partition}) obtenues lors de l'expérience de simulation. Le niveau sonore de sortie est identique pour tous les sujets.

Tous les sujets réalisent l'expérience simultanément, dans un environnement calme. Ils n'ont pas le droit de s'adresser la parole pendant l'expérience.

Un expérimentateur est présent durant la totalité de l'expérience, afin de contrôler le bon déroulement de cette dernière, et de répondre à d'éventuelles questions des sujets. 

\textbf{Participants}

10 étudiants (2 femmes) de L’École Centrale de Nantes ont participé à l'expérience. Aucun d'entre eux n'a réalisé l'expérience de simulation. Tous les sujets ont un peu près le même âge (moyenne: 23.1, écart-type: 1.8). Tous les sujets ont vécu dans la même ville (Nantes), au minimum pendant les deux dernières années précédant l'expérience.

Tous les sujets ont réalisés l'expérience avec succès.

\subsection{Données et méthodes d'analyses}

\subsubsection{Nature des données}

\subsubsection{Outils statistiques}

\subsection{Validité écologique de l'expérience}

\subsection{Vérification de l'agrément des scènes simulées}

\subsection{Analyse des descripteurs structurels}

\subsection{Analyse des descripteurs sémantiques}

\subsubsection{Étude qualitatives de la répartition des classes}

\subsubsection{L'émergence de marqueurs sonores}

\subsubsection{Étude quantitative de la répartition des classes}

\subsection{L'influence des marqueurs sonores sur l'agrément}

\subsection{Discussions}

\section{Simulation et approche positive}

\subsection{Objectif de l'expérience}

\subsection{Planification expérimentale}

\subsection{Données et méthodes d'analyses}

\subsection{Discussions}

\section{Simulation et approche catégorielle}

\subsection{Objectif de l'expérience}

\subsection{Planification expérimentale}

\subsection{Données et méthodes d'analyses}

\subsection{Discussions}
%*****************************************
%*****************************************
%*****************************************
%*****************************************
%*****************************************




