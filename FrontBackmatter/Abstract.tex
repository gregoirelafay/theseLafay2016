%*******************************************************
% Abstract
%*******************************************************
%\renewcommand{\abstractname}{Abstract}
\pdfbookmark[1]{Abstract}{Abstract}
\begingroup
\let\clearpage\relax
\let\cleardoublepage\relax
\let\cleardoublepage\relax

\chapter*{Abstract}
TODO

\vfill


\pdfbookmark[1]{Zusammenfassung}{Zusammenfassung}
\chapter*{Résumé}

Le présent travail de thèse porte sur l'analyse des scènes sonores. 

Par scène sonore, on entend ici un extrait d'un environnement sonore, le résultat auditif du mélange de différentes sources émettrices. 

La notion d'environnement ne restreint pas la nature des sources considérées, nous éloignant de fait des champs de recherche portant spécifiquement sur la parole ou la musique. 

L'étude est pluridisciplinaire. Elle traite à la fois de l'analyse sensorielle, des mécanismes mises en œuvre par le système auditif afin d'acquérir, organiser et utiliser l'information, et de l'analyse automatique, des algorithmes ayant pour but de mimer ces mécanismes. Elle s'inscrit alors complètement dans le champs des sciences cognitives, portant à la fois sur la perception et l'intelligence artificielle. 

De part sa diversité, l'environnement sonore est un objet complexe. De plus, son analyse, le fait d'en extraire du sens, est un processus qui relève à la fois de l'information contenue dans les données, mais également du contexte. 

Comprendre comment capter l'information utile dans un tel objet demande de se placer dans un cadre écologique, de privilégier l'utilisation de données réels, enregistrés, aux données de synthèses. 

Toute expérience suppose une certaine emprise de la part de l'expérimentateur sur la nature des stimuli considérés. Or dans les cas des environnements sonores, peu de travaux portent sur des moyens permettant au chercheur de faire varier de manière contrôlée les caractéristiques des stimuli.

Conscient de cette problématique, nous proposons ici un modèle permettant de simuler, à partir d'enregistrements de sons isolés, des scènes sonores dont nous contrôlons les propriétés structurelles, à savoir, l'intensité, la densité et la diversité des sources sonores en présence. Le modèle envisage la scène sonore comme un objet composite, une somme de sons sources. Cette vision hétéroclite est motivée par les connaissances disponibles sur le système auditif humain.

Fort de ce modèle, nous choisissons trois cas d'application. Le premier porte sur la perception et questionne la notion d'agrément perçu dans des environnements urbains. Le deuxième porte sur la détection automatique des événements sonores et propose une méthodologie afin d'évaluer les performances des algorithmes. Ces deux premiers cas profitent directement du processus de génération de stimuli proposé. Le troisième porte le recouvrement automatique des similarités entre scènes sonores, et s'appuie sur la vision hétéroclite des scènes afin de proposer un algorithme adapté.

\endgroup			

\vfill