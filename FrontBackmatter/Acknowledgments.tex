%*******************************************************
% Acknowledgments
%*******************************************************
\pdfbookmark[1]{Remerciements}{Remerciements}

\medskip

\begingroup
\let\clearpage\relax
\let\cleardoublepage\relax
\let\cleardoublepage\relax
\chapter*{Remerciements}

Je remercie tout d'abord très chaleureusement les membres du jury, Bertrand David et Catherine Lavandier, rapporteurs, Jean-julien Aucouturier, examinateur, et Alain de Cheveigné, examinateur et président du jury, pour l’intérêt qu'ils ont porté à mes travaux, et leurs retours précieux pour la finalisation du présent document. \\

Pour la grande qualité de leur encadrement, je remercie non moins chaleureusement mes encadrants : Jean-François Lafay, sans qui cette thèse n'aurait pas pu démarrer, Jérôme Idier, qui a accepté de diriger cette thèse, et qui, en outre, a pris le temps d'expliquer au jeune étudiant que j'étais les réalités du monde de la recherche, Jean-François Petiot, qui m'a apporté toute son expertise dans le domaine de l'analyse sensorielle. Je lui suis, par ailleurs, infiniment reconnaissant de la confiance qu'il m'a accordée en m'offrant de monter, puis de donner, à Central Nantes, des cours ayant trait à mes sujets de recherche. L'expérience, qui a débuté il y a trois ans, et qui se poursuit encore, aura été enrichissante, et plus que cela sans doute. 

Il est enfin des rencontres qui changent le cours d'une vie. Celle avec Mathieu Lagrange a été décisive, tant du point de vue scientifique que du point de vue personnel. Pour la liberté qu'il m'a laissée dans la façon d'aborder mon sujet, non sans me canaliser quand, par enthousiasme, je m'égarais, pour son investissement et sa motivation sans faille, pour la richesse de nos débats (tant de visions et d'intuitions partagées), pour m'avoir initié au \textit{Machine-Listening} et, d'une manière générale, pour tout ce qu'il m'a appris, je lui témoigne ici ma profonde gratitude, et lui dit, je le peux bien aujourd'hui, toute mon affection. \\

Pour leur accueil et nos discussions autour de thèmes scientifiques connexes, je remercie les membres de l’équipe Analyse et Décision en Traitement du Signal du laboratoire IRCCyN, Marie-Françoise Lucas, Saïd Moussaoui, Sébastien Bourguignon et Éric Le Carpentier. \\

Pour tous nos échanges durant ces trois années, je remercie mes collègues doctorants, en particulier Robin Tournemenne, Corentin Friedrich, Mohamed-Anis Dhuieb, Maxime Sirbu et Jean-Remy Gloaguen. \\

Pour notre collaboration autour de la transformée en \textit{scattering}, je remercie tout spécialement Vincent Lostanlen et Joakim Andén. \\

Pour avoir permis d'inscrire mes travaux sur l'évaluation des algorithmes en \textit{Machine-Listening} dans le cadre du challenge international DCASE, et pour nos multiples collaborations scientifiques, je remercie vivement Emmanouil Benetos. \\

Pour m'avoir enseigné les bases de l'analyse des signaux audio, et également pour m'avoir donner le goût de la pluridisciplinarité, je remercie l'équipe pédagogique du master ATIAM. Notamment, les membres de l'équipe Perception et Design Sonore de l'IRCAM qui m'ont accueilli lors de mon stage de Master. Un salut tout particulier à Nicolas Misdariis, pour son excellent encadrement, et pour avoir été celui qui m'a fait découvrir la psychologie cognitive. \\

Pour être le premier à avoir élaboré l'outil de simulation de scènes sonores, pour son expertise scientifique et son expertise en programmation, pour ses conseils avisés et son aide de tous les instants, mais également, pour son amitié, je remercie chaudement Mathias Rossignol. Cela fait maintenant quatre ans que nous collaborons, et je suis heureux que l'expérience se poursuive. \\

Pour leur motivation, leur bonne humeur et leur intérêt, je tiens à remercier les élèves de l'École Centrale de Nantes que j'ai eu le plaisir d'encadrer au cours de nombreux projets étudiants, Simon Dubois, Adrien Urso, Nicolas Dany, Jean-Baptiste Kaiser, Thomas Forgue, Lucas Sanchez, César Lacroix, Théo Cordoliani, Hugo Pagnier, Baptiste Parquier, Oriane Cosson et Pablo Bonachela-Guhmann. Ces projets ont permis d'explorer plusieurs pistes de recherches intéressantes, et certains ont par ailleurs directement participé à la mise en place des protocoles expérimentaux présentés dans cette thèse. \\

Pour avoir accepté d'être mes «~cobayes~», je remercie grandement les très nombreux étudiants de l'École Centrale de Nantes qui ont participé aux non moins nombreuses expériences perceptives réalisées dans la cadre de cette thèse. Inutile de dire que ce travail n'aurait pas été possible sans leur participation volontaire et motivée.

Pour leur soutien inconditionnel, j'adresse encore mes remerciements à ma famille, mes grands parents, ma sœur et bien sûr mes parents, à qui je dois beaucoup. \\

Enfin, j'embrasse Camille qui fut si loin et à la fois si proche durant ces trois années.


\endgroup



